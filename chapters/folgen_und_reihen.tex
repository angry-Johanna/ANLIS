\documentclass[../main.tex]{subfiles}

\begin{document}
%%%%%%%%%%%%%%%%%%%%%
%                   %
% FOLGEN UND REIHEN %
%                   %
%%%%%%%%%%%%%%%%%%%%%
\chapter{SW02 Folgen und Reihen}
\section{Arithmetische Folgen und Reihen}
$a_n=a_1+(n-1)\cdot d$ \\
$(a_n) = a_1,a_2,a_3,...,a_n,...$

\begin{flushleft}
Differenz d zweier beliebiger aufeinanderfolgender Glieder $a_n,a_{n+1}$ ist konstant.
\end{flushleft}

Eine AF ist eindeutig beschrieben durch zwei Grössen:
\begin{itemize}
    \item beliebiges Glied $a_n$ und Differenz $d$
    \item zwei beliebige Glieder $a_n$ und $a_{n+k}$
\end{itemize}

\begin{flushleft}
\textbf{Bildungsgesetz}: Funktionsvorschrift nach welcher aus $n$ das $n$-Glied ($a_n$) berechnet werden kann.
\end{flushleft}

\subsection{Beispiele von Folgen}
$(a_n) = -\frac{1}{2},-\frac{1}{4},-\frac{1}{8},...$ Bildungsgesetz: $a_n=-\frac{1}{2n}$ \\ [7pt]
$(a_n) = 1^3,2^3,3^3,...$ Bildungsgesetz: $a_n=n^3$ \\ [7pt]
$(a_n) = 0,\frac{1}{2},\frac{2}{3},\frac{3}{4},...$ Bildungsgesetz: $a_n=\frac{n-1}{n}$

\subsection{Summe der Glieder einer AF: Arithmetische Reihe}
$\sum\limits_{k=1}^n a_k = na_1 + d \frac{n(n-1)}{2} = n \frac{a_1 + a_n}{2}$\\ [7pt]
Wobei bei "$n \frac{a_1 + a_n}{2}$" $a_1$ das erste Glied ist, $a_n$ das letzte, $n$ die Anzahl Glieder und $2$ den Mittelwert vom ersten und letzten Glied bildet.

\subsection{Nützliche andere Formeln}

\begin{tabularx}{1\textwidth} { 
    >{\centering\arraybackslash}X 
    >{\centering\arraybackslash}X  }
    Gegeben: $a_n = v$, $a_{n+x} = z$
    &
    Gesucht $d$: $d = \frac{z - v}{(n+x) -n}$
    \\ [7pt]
    \begin{math}
        {}
    \end{math}
    &
    \begin{math}
        {}
    \end{math}
    \\ [7pt]
\end{tabularx}


\section{Geometrische Folgen}
$a_n=a_1\cdot q^{n-1}$
\begin{flushleft}
    Die geometrische Folge ist dadurch charakterisiert, dass der Quotient q zweier beliebiger aufeinanderfolgender Glieder $a_n$ und $a_{n+1}$ konstant ist.
\end{flushleft}

\begin{tabularx}{0.8\textwidth} { 
    >{\centering\arraybackslash}X 
    >{\centering\arraybackslash}X  }
    \begin{math}
        a_{n+1} = qa_n, n=1,2
    \end{math}
    &
    \begin{math}
        q = \frac{a_{n+1}}{a_n}
    \end{math}
    \\ [7pt]
\end{tabularx}

Eine GF ist eindeutig beschrieben durch zwei Grössen, entweder:
\begin{itemize}
    \item durch ein beliebiges Glied $a_n$ und den Quotienten $q$
    \item durch zwei beliebige Glieder $a_n$ und $a_{n+k}$
\end{itemize}

\section{Geometrische Reihe}
$S_n=\sum\limits_{i=1}^na_i=a_1\cdot\frac{q^n-1}{q-1}$

\section{Rechnen mit Folgen, Eigenschaften}
\begin{itemize}
    \item Folge $(a_n)$ multipliziert man mit einer reellen Zahl $\lambda$, indem man jedes Glied der Folge mit dieser Zahl multipliziert: \\ [7pt]
    $\lambda(a_n) = (\lambda a_n)$
    \item Zwei Folgen $(a_n)$ und $(b_n)$ addiert man, indem man entsprechende Glieder addiert: \\ [7pt]
    $(a_n)+(b_n) = (a_n + b_n)$
    \item Eine Folge heisst \textbf{
        konstante Folge}, falls $a_n = c \in \mathbb{R}, \forall n \in \mathbb{N}$ \\
    AF ist konstant wenn $d=0$, \\
    GF ist konstant wenn $q=1$
    \item Eine Folge $(a_n)$ ist \textbf{streng monoton zunehmend/abnehmend} falls $(a_{n+1} > a_n)$ bzw $(a_{n+1} < a_n)$
    \item Eine Folge $(a_n)$ ist \textbf{beschränkt} (höhö) falls eine positive Zahl $c$ existiert mit $|a_n|\leq c, \forall n$: alle Glieder der Folge liegen im Graphen unter einem Teppich der Breite $2c$. Anderfalls heisst die Folge $(a_n)$ \textbf{unbeschränkt}
\end{itemize}


\section{Folgen Eigenschaften}
\subsection{Konvergenz/Divergenz}
Folge ist \textbf{konvergent}, wenn sie einen Grenzwert besitzt. Folge ist \textbf{divergent}, wenn es keinen 
Grenzwert gibt.

\subsection{Explizite Folge}
Beispiel: $(a_n)_n=(\frac{1}{n})_n$ \\
$a_n$ für den Wert direkt erkennbar.

\subsection{Rekursive Folge}
Beispiel: $a_{n+1}=a_n+a_{n+1}$ \\
$a_n$ ergibt sich aus den vorherigen Glieder der Rekursionsvorschrift.

\subsection{Nullfolge}
Folge konvergiert zum Grenzwert 0.

\subsection{Alternierende Folge}
Beinhaltet etwas in der Art $(-1)^n$. Fallunterscheidung machen für GW, weil wenn $n$ gerade, dann Folgenglied
positiv, wenn $n$ ungerade, dann Folgenglied negativ.



\end{document}