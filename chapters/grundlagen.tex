\documentclass[../main.tex]{subfiles}

\begin{document}
%%%%%%%%%%%%%%
%            %
% GRUNDLAGEN %
%            %
%%%%%%%%%%%%%%
\chapter{Grundlagen}

%%%%%%%%%%%
% WURZELN %
%%%%%%%%%%%
\section{Wurzeln}

\begin{tabularx}{1\textwidth} { 
     >{\centering\arraybackslash}X 
     >{\centering\arraybackslash}X  }
    \begin{math}
        \sqrt{x} =  x^\frac{1}{2}
    \end{math}
    &
    \begin{math}
        \sqrt[a]{x^b} = x^\frac{b}{a}
    \end{math}
    \\ [7pt]
    \begin{math}
        \sqrt{a \times b} = \sqrt{a} \times \sqrt{b}
    \end{math}
    &
    \begin{math}
      \sqrt{\frac{a}{b}} = \frac{\sqrt{a}}{\sqrt{b}}
    \end{math}
    \\ [7pt]
    \begin{math}
        \sqrt{a} + \sqrt{b} \ne  \sqrt{a+b}
    \end{math}
    &
    \begin{math}
        \sqrt{a} - \sqrt{b} \ne  \sqrt{a-b}
    \end{math}
    \\ [7pt]
    \begin{math}
        \sqrt{a^2 \times b} = a \times \sqrt{b}
    \end{math}
    &
    \begin{math}
        \frac{a}{\sqrt{b}} = \frac{a \sqrt{b}}{b}
    \end{math}
    \\ [7pt]
    \begin{math}
        \sqrt[b]{a^b} = ({a^b})^{\frac{1}{b}} = a
    \end{math}
    &
    \begin{math}
        \frac{1}{\sqrt[n]{a}} = a^{-\frac{1}{n}}
    \end{math}
    \\ [7pt]
    \begin{math}
        {}
    \end{math}
    &
    \begin{math}
        {}
    \end{math}
    \\ [7pt]
\end{tabularx}

%%%%%%%%%%%%
% POTENZEN %
%%%%%%%%%%%%
\section{Potenzen}

\begin{tabularx}{1\textwidth} { 
    >{\centering\arraybackslash}X 
    >{\centering\arraybackslash}X  }
    \begin{math}
        x^{-a} = \frac{1}{x^a}
    \end{math}
    &
    \begin{math}
        \frac{a}{bx^{-c}} = \frac{a}{b}x^{-c}
    \end{math}
    \\ [7pt]
    \begin{math}
        x^a \times{x^b} = x^{a+b}
    \end{math}
    &
    \begin{math}
        \frac{x^a}{x^b} = x^{a-b}
    \end{math}
    \\ [7pt]
    \begin{math}
        {x^a}^b = x^{a \times{b}}
    \end{math}
    &
    \begin{math}
        \frac{a^x}{a^{x+1}} = \frac{1}{a}
    \end{math}
    \\ [7pt]
    \begin{math}
        {}
    \end{math}
    &
    \begin{math}
        {}
    \end{math}
    \\ [7pt]
    \begin{math}
        {}
    \end{math}
    &
    \begin{math}
        {}
    \end{math}
    \\ [7pt]
\end{tabularx}


%%%%%%%%%%
% BRÜCHE %
%%%%%%%%%%
\section{Brüche}

\begin{tabularx}{1\textwidth} { 
    >{\centering\arraybackslash}X 
    >{\centering\arraybackslash}X  }
    \begin{math}
        \frac{a}{b} + \frac{c}{d} = \frac{ad}{bd} + \frac{cb}{bd} = \frac{ab+cb}{bd}
    \end{math}
    &
    \begin{math}
        \frac{a}{b} - \frac{c}{d} = \frac{ad}{bd} - \frac{cb}{bd} = \frac{ab-cb}{bd}
    \end{math}
    \\ [7pt]
    \begin{math}
        \frac{a}{b} \times \frac{c}{d} = \frac{ac}{bd}
    \end{math}
    &
    \begin{math}
       \frac{\frac{a}{b}}{\frac{c}{d}} = \frac{a}{b} \times \frac{d}{c}
    \end{math}
    \\ [7pt]
    \begin{math}
        \frac{1}{x} = x^{-1}
    \end{math}
    &
    \begin{math}
        \frac{1}{x^2} = x^{-2}
    \end{math}
    \\ [7pt]
    \begin{math}
        \frac{1}{x^3} = x^{-3}
    \end{math}
    &
    \begin{math}
        \frac{4}{3}x^{-4} = \frac{4}{3x^{-4}}
    \end{math}
    \\ [7pt]
    \begin{math}
        \frac{x}{5} = \frac{1}{5}x
    \end{math}
    &
    \begin{math}
        \frac{x^4}{9} = \frac{1}{9}x^4
    \end{math}
    \\ [7pt]
    \begin{math}
        {}
    \end{math}
    &
    \begin{math}
        {}
    \end{math}
    \\ [7pt]
    \begin{math}
        {}
    \end{math}
    &
    \begin{math}
        {}
    \end{math}
    \\ [7pt]
\end{tabularx}


%%%%%%%%%%%%%%%
% LOGARITHMEN %
%%%%%%%%%%%%%%%
\section{Logarithmen}

\begin{tabularx}{1\textwidth} { 
    >{\centering\arraybackslash}X 
    >{\centering\arraybackslash}X  }
    \begin{math}
        y = log_a(x) <=> x = a^y
    \end{math}
    &
    \begin{math}
        \log_{b}(xy) = \log_{b}(x) + \log_{b}(y)
    \end{math}
    \\ [7pt]
    \begin{math}
        \log_{b}(\frac{x}{y}) = \log_{b}(x) - \log_{b}(y)
    \end{math}
    &
    \begin{math}
        \log_{b}(x^y) = y \log_{b}(x)
    \end{math}
    \\ [7pt]
    \begin{math}
        {}
    \end{math}
    &
    \begin{math}
        {}
    \end{math}
    \\ [7pt]
\end{tabularx}

%%%%%%%%%%
% BINOME %
%%%%%%%%%%
\section{Binome}
\subsection{1. Binom}
\[(a+b)^2 = a^2 + 2ab + b^2\]

\subsection{2. Binom}
\[(a-b)^2 = a^2 - 2ab + b^2\]

\subsection{3. Binom}
\[(a+b)(a-b) = a^2 -b^2\]

%%%%%%%%%%%%%%%%%%%%%%%%%%
% QUADRATISCHE GLEICHUNG %
%%%%%%%%%%%%%%%%%%%%%%%%%%
\section{Quadratische Gleichung}
Für:
\[ax^2+bx+c=0\]
Dann:
\[x_{1,2}=\frac{-b\pm\sqrt{b^2-4ac}}{2a}\]

%%%%%%%%%%%%%%%
% ABLEITUNGEN %
%%%%%%%%%%%%%%%
\section{Ableitungen/Integrationen}
Wenn integrieren, $+C$ nicht vergessen! \\ [7pt]
\begin{tabularx}{0.8\textwidth} { 
    >{\centering\arraybackslash}X 
    >{\centering\arraybackslash}X  }
    \hline
    f(x) & f'(x) \\ [7pt]
    \hline \\
    $x$ & $1$
    \\ [7pt]
    $x^a$ & $ax^{a-1}$
    \\ [7pt]
    $\frac{x^{a+1}}{a+1}$ & $x^a$
    \\ [7pt]
    $\sqrt[n]{x^m} = x^{\frac{m}{n}}$ & $\frac{m}{n}x^{\frac{m}{n}-1}, a \neq -1$
    \\ [7pt]
    $e^x$ & $e^x$
    \\ [7pt]
    $a^x$ & $(ln(a))a^x (a<0)$
    \\ [7pt]
    $\frac{a^x}{ln(a)}$ & $a^x$
    \\ [7pt]
    $ln(x)$ & $\frac{1}{x}$ 
    \\ [7pt]
    $a \times ln(x)$ & $\frac{a}{x}$
    \\ [7pt]
    $\sin x$ & $\cos x$
    \\ [7pt]
    $\cos x$ & $-\sin x$
    \\ [7pt]
    $\tan x$ & $\frac{1}{\cos ^2x} = 1 + tan ^2 x$
    \\ [7pt]
    $-\cot x$ & $\frac{1}{\sin ^2x}$
    \\ [7pt]
    $\arcsin x$ & $\frac{1}{\sqrt[]{1-x^2}}$ +
    \\ [7pt]
    $-\arcsin x$ & $\frac{1}{\sqrt[]{1-x^2}}$
    \\ [7pt]
    $\arccos x$ & $-\frac{1}{\sqrt{1 - x2}}$
    \\ [7pt]
    $\arctan x$ & $\frac{1}{1 + x^2}$ 
    \\ [7pt]
    $-\arctan x$ & $\frac{1}{1 + x^2}$ 
    \\ [7pt]
    $\sinh x$ & $\cosh x$
    \\ [7pt]
    $\cosh x$ & $\sinh x$
    \\ [7pt]
    $\tanh x$ & $\frac{1}{\cosh ^2x} = 1 + tanh ^2 x$
    \\ [7pt]
    $\operatorname{arsinh} x$ & $\frac{1}{\sqrt{1 + x2}}$
    \\ [7pt]
    $\operatorname{arcosh} x$ & $\frac{1}{\sqrt{1 - x2}}$
    \\ [7pt]
    $\operatorname{artanh} x$ & $\frac{1}{1 - x^2}$ 
    \\ [7pt]
    $\coth x$ & $-\frac{1}{\sinh^2x}$
\end{tabularx} \\ [7pt]
* falls $x \in (-1,1)$


%%%%%%%%%%%%%
% BEISPIELE %
%%%%%%%%%%%%%
\section{Beispiele}

\begin{tabularx}{1\textwidth} { 
    >{\centering\arraybackslash}X 
    >{\centering\arraybackslash}X  }
    \begin{math}
       \frac{2}{3\sqrt[4]{x^5}} = \frac{2}{3x^{-\frac{5}{4}}} = \frac{2}{3}x^{-\frac{5}{4}}
    \end{math}
    &
    \begin{math}
        {}
    \end{math}
    \\ [7pt]
    \begin{math}
        {}
    \end{math}
    &
    \begin{math}
        {}
    \end{math}
    \\ [7pt]
\end{tabularx}


\end{document}