\documentclass[../main.tex]{subfiles}

\begin{document}
%%%%%%%%%%%%%%%%%%%%%%%%%%%%%%%%%%%%%%%%%%%%%%%%%%%%%%%%
%                                                      %
% Integralrechnung I -- Flächenberechnung und Integral %
%                                                      %
%%%%%%%%%%%%%%%%%%%%%%%%%%%%%%%%%%%%%%%%%%%%%%%%%%%%%%%%

\chapter{SW08 Integralrechnung I -- Flächenberechnung und Integral}
Umkehrung der Differenzierung / Ableitung

\section{Stammfunktion}
Eine differenzierbare Funktion $F(x)$ heisst Stammfunktion von $f(x)$ falls: \\
$F'(x) = f(x)$ \\ [7pt]
Eigenschaften der Stammfunktion:
\begin{itemize}
    \item Zu jeder stetigen Funktion $f(x)$ gibt es $\infty-$viele Stammfunktionen
    \item Zwei beliebige Stammfunktionen $F_1(x)$ und $F_2(x)$ unterscheiden sich nur durch eine additive Konstante, dh \\
    $F_1(x) - F_2(x) = const$
    \item Ist $F_1(x)$ eine beliebige Stammfunktion von $f(x)$, dann ist auch $F_2(x) = F_1(x) + C (C \in \mathbb{R})$ 
    eine Stammfunktion von $f(x)$. Daher ist die Menge aller Stammfunktionen von der Form \\
    $F(x) = F_1(x) + C$, wobei C eine beliebige (reelle) Konstante ist.
\end{itemize}

\section{Umkehrung der Differentiation}
Für Polynomfunktion: \\
$f(x)=x^n \to  F(x)=\frac{x^{n+1}}{n+1}+C$ \\ [7pt]
Für alle anderen Funktionen siehe: \hyperref[sec:Ableitungen]{\ref{sec:Ableitungen} \nameref{sec:Ableitungen}} \\
Konstante $+ C$ dabei nicht vergessen!

\section{Bestimmtes Integral Flächenberechnung}
$f(x), [a,b]$ \\ [7pt]
$I = \int\limits_a^bf(x)dx = \lim\limits_{n \to \infty} \sum\limits^n_{k=1}f(x_k)\Delta x$ \\ [7pt]
$\Delta x = \frac{b-a}{n}$ \\ [7pt]
$x_k = a + k \Delta x$ \\ [7pt]
Wenn rechter Rand: $f$ an der Stelle $x^*_k = x_k$ \\ [7pt]
Wenn linker Rand: $f$ an der Stelle $x^*_k = x_{k-1}$ \\ [7pt]
$S_n = \sum\limits^n_{k=1}f(x_k)\Delta x$ auflösen bis alle $k$ weg (siehe \hyperref[sec:Summen_vereinfachen]{\ref{sec:Summen_vereinfachen} \nameref{sec:Summen_vereinfachen}}) \\ [7pt]
$\lim\limits_{n \to \infty} S_n$ auflösen, Resultat gleich Fläche im Interval $[a,b]$

\subsection{Beispiel Rechter Rand}
(siehe \hyperref[sec:Summen_vereinfachen]{\ref{sec:Summen_vereinfachen} \nameref{sec:Summen_vereinfachen}}) \\
$y = x^2, [0,1], a=0, b=1$ \\ [7pt]
$\Delta x = \frac{b-a}{n} = \frac{1-0}{n} = \frac{1}{n}$ \\ [7pt]
$x_k = a + k \Delta x = 0 + k \frac{1}{n} = \frac{k}{n}$ \\ [7pt]
\textbf{Rechter Rand:} $x^*_k = x_k, f(x^*_k)=f(x_k)=x^2_k = (\frac{k}{n})^2$ \\ [7pt]
$S_n = \sum\limits^n_{k=1}f(x_k)\Delta x = \sum\limits^n_{k=1} (\frac{k}{n})^2 \frac{1}{n} 
= \sum\limits^n_{k=1} \frac{k^2}{n^3}
= \frac{1}{n^3} \sum\limits^n_{k=1} k^2$ \\ [7pt]
$ = \frac{1}{n^3} \frac{n(n+1)(2n+1)}{6} = ... = \frac{1}{6}(1 + \frac{1}{n})(2 + \frac{1}{n})$ \\ [7pt]
$\lim\limits_{n \to \infty} S_n = \lim\limits_{n \to \infty} \frac{1}{6}(1 + \frac{1}{n})(2 + \frac{1}{n})
= \frac{1}{6} \lim\limits_{n \to \infty} (1 + \frac{1}{n})(2 + \frac{1}{n}) = \frac{1}{3}$


\subsection{Beispiel Linker Rand}
(siehe \hyperref[sec:Summen_vereinfachen]{\ref{sec:Summen_vereinfachen} \nameref{sec:Summen_vereinfachen}}) \\
$y = x^3, [0,2], a=0, b=2$ \\ [7pt]
$\Delta x = \frac{b-a}{n} = \frac{2-0}{n} = \frac{2}{n}$ \\ [7pt]
$x_k = a + k \Delta x = 0 + k \frac{2}{n} = \frac{2k}{n}$ \\ [7pt]
\textbf{Linker Rand:} $x^*_k = x_{k-1}, f(x^*_k)=f(x_{k-1})=x^3_{k-1}=(\frac{2(k-1)}{n})^3$ \\ [7pt]
$S_n = \sum\limits^n_{k=1}f(x_k)\Delta x = \sum\limits^n_{k=1} (\frac{2(k-1)}{n})^3 \frac{2}{n}
= \sum\limits^n_{k=1} (\frac{2}{n})^3 (k-1)^3 \frac{2}{n}
= \sum\limits^n_{k=1} (\frac{2}{n})^4 (k-1)^3 $ \\ [7pt]
$(\frac{2}{n})^4 \sum\limits^n_{k=1} (k-1)^3
= (\frac{2}{n})^4 \sum\limits^{n -1}_{k=1} k^3 
= (\frac{2}{n})^4 \frac{(n(n-1))^2}{n^4} = 4(1-\frac{1}{n})^2$ \\ [7pt]
$\lim\limits_{n \to \infty} S_n = \lim\limits_{n \to \infty} 4(1-\frac{1}{n})^2 = 4 \lim\limits_{n \to \infty} (1-\frac{1}{n})^2 = 4$

\section{Summen vereinfachen}
\label{sec:Summen_vereinfachen}
$\sum\limits^n_{k=1} k = \frac{n(n+1)}{2}$ \\ [7pt]
$\sum\limits^n_{k=1} k^2 = \frac{n(n+1)(2n+1)}{6}$ \\ [7pt]
$\sum\limits^n_{k=1} k^3 = (\frac{n(n+1)}{2})^2$ \\ [7pt]
$\sum\limits^n_{k=1} (k-1)^3 = \sum\limits^{\textbf{n-1}}_{k=1} k^3 = (\frac{n(n-1)}{2})^2$



\end{document}