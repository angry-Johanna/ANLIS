\documentclass[../main.tex]{subfiles}

\begin{document}
%%%%%%%%%%%%%%
%            %
% FUNKTIONEN %
%            %
%%%%%%%%%%%%%%

\chapter{SW01 Funktionen}
\section{Lineare Funktion}
\[f(x)=ax+b\]
a = Steigung

\section{Polynomfunktion}
Grad der Funktion: Höchster Exponent von x. \\
Nullstellen: Maximal so viele wie der Grad der Funktion. 
\[f(x)=ax^n+bx^{n-1}+cx^{n-2}...\]

\section{Quadratische Funktionen}
Polynomfunktion zweites Grades
\[f(x)=ax^2+bx+c\]

\section{Exponentialfunktion}
\[f(x)=a \times b^x\]


\section{Logarithmusfunktion}
Umkehrfunktion von Exponentialfunktion
\[f(x)=log_b(x)\]


\end{document}